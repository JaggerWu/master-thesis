\chapter{Rigid Body Dynamics Simulation}

This chapter mainly introduces rigid body simulation to help you understand how computer simulate rigid dynamics based on traditional newton-euler equations. For more details, some contact forces solvers are decribed in this chapter. Afterwards, we will use one of solver to run some simulation and get the image data for the next step, grids-transfer. All the discussion about rigid simulation and contacts solver are based on 2-$D$ view.

\section{Rigid dynamics Simulation}

\subsection{Classical mechanics}
Simulation of the motion of a system of rigid bodies is based on a famous system of differential equations, the \textit{Newton–Euler equations}, which can be derived from Newton’s laws and other basic concepts from classical mechanics:

\begin{enumerate}
    \item Newton’s first law: The velocity of a body remains unchanged r unless acted upon by a force.
    \item Newton’s second law: The time rate of change of momentum of r a body is equal to the applied force.
    \item Newton’s third law: For every force, there is an equal and opposite force.
\end{enumerate}

Before presenting the Newton–Euler equations, we need to introduce a number of concepts from classical mechanics. We will start with one simple simulation with only position vector $x(t)$ and velocity vector $\pmb{v}$. Then, we will introduce some concepts by adding rotatiopn to pur simulation, rotational velocity $\pmb{\omega}$, and moment $\pmb{\tau}$ (also known as a torque.


\subsection{Simulation Basics}

Firstly, we can start with a simple simulation with only position and velocity. Simulating the motion of a rigid body is almost the same as simulating the motion of a particle, so I will start with partcle simulation. For particle simulation, we let function $\pmb{p}(t)$ describe the particle's location in world space at time $t$. Then we use $\pmb{v}(t)=\frac{\dif}{\dif(t)}\pmb{p}(t)$
to denote the velocity of the particle at time t. So, the state of a particle at a time t is the particle's position and velocity. We generalize this concept by defining a state vector $\textbf{Y}(t)$ for a system: for a single particle,
\begin{equation}
    \textbf{Y}(t) = \left(
        \begin{array}{c}
            \pmb{p}_{1}(t) \\
            \pmb{v}_{1}(t) \\
        \end{array}
    \right)
\end{equation}

For a system with $n$ particles, we enlarge $\textbf{Y}(t)$ to be
\begin{equation}
    \textbf{Y}(t) = \left(
    \begin{array}{c}
        \pmb{p}_{1}(t) \\
        \pmb{v}_{1}(t) \\
        ... \\
        \pmb{p}_{n}(t) \\
        \pmb{v}_{n}(t) \\
    \end{array}
    \right)
\end{equation}

However, to simulate the motion of particles actually, we need to know one more thing -- the forces. $\pmb{f}(t)$ is defined as the force acting on the particle. If the mass of the particle is $m$, then the changes of $\textbf{Y}(t)$ will be given by
\begin{equation}
    \frac{\dif}{\dif(t)}\textbf{Y}(t) = \frac{\dif}{\dif(t)}\left(
        \begin{array}{c}
            \pmb{p}(t) \\
            \pmb{v}(t) \\
        \end{array} \right) = \left(
        \begin{array}{c}
            \pmb{v}(t) \\
            \pmb{f}(t)/m \\
        \end{array} \right)
\end{equation}

\subsection{Rigid Body Concepts}
Unlike a particle, a rigid body occupies a volume of space and has a particular shape. Rigid bodies are more complicated, beside translating them, we can rotate them as well. To locate a rigid body, we use $\pmb{p}(t)$ to denote their translation and a rotation matrix $\pmb{R}(t)$ to describe their rotation.

\subsection{Rigid Body Equations of Motions}
 Whereas linear momentum $\pmb{P}(t)$ is related to linear velocity with a scalar (the mass), angular momentum is related to angular velocity with a matrix \(\pmb{I}\), called the angular inertia matrix. The reason for this is that objects generally have different angular inertias around different axes of rotation. Angular momentum is defined as \(\pmb{L}\). The linear momentum is defined as \ref{lm}, and angular momentum is defined as \ref{am}
\begin{equation}
    m \dot{\pmb{v}(t)} = \pmb{f}(t)
    \label{force}
\end{equation}
\begin{equation}
    \pmb{P}(t) = m \pmb{v} (t)
    \label{lm}
\end{equation}

\begin{equation}
    \pmb{L}(t) = \pmb{I}(t)\pmb{\omega}(t)
    \label{am}
\end{equation}

The total torque $\pmb{\tau}$ applied to the body is equal to the rate of change of the angular momentum, as defined in \ref{tau}:
\begin{equation}
    \pmb{\tau} = \frac{\dif}{\dif t }\pmb{L} = \frac{\dif}{\dif t }(\pmb{I}\pmb{\omega})
    \label{tau}
\end{equation}

 Then we can covert all concepts we need to define stare $\textbf{Y}$ for a rigid body.

\begin{equation}
    \textbf{Y}(t) = \left(
        \begin{array}{c}
            \pmb{p}(t) \\
            \pmb{R}(t) \\
            \pmb{P}(t) \\
            \pmb{L}(t) \\
        \end{array}
    \right)
\end{equation}

Like what is epressed in $\textbf{Y}(t)$, the state of a rigid body is mainly consist by its position and orientation (describing spatial information), and its linear and angualr momentum(describe velocity information). Since mass $m$ and bodyspace inertia tensor $\pmb{I}_{body}$ are constants, we can the auxiliary quantities $\pmb{I}(t)$, $\pmb{\omega}(t)$ at any given time.

\begin{equation}
    \pmb{v}(t) = \frac{\pmb{P}(t)}{m} \quad
    \pmb{I}(t) = \pmb{R}(t)\pmb{I}_{body}\pmb{R}(t)^{T} \quad
    \pmb{\omega}(t) = \pmb{I}(t)^{-1}\pmb{L}(t)
\end{equation}

The derivative $\frac{\dif}{\dif t}\textbf{Y}(t)$ is
\begin{equation}
    \frac{\dif}{\dif{t}}\textbf{Y}(t) = \frac{\dif}{\dif t}\left(
        \begin{array}{c}
            \pmb{p}(t) \\
            \pmb{R}(t) \\
            m \pmb{v}(t) \\
            \pmb{L}(t) \\
        \end{array}
    \right) = \left(
        \begin{array}{c}
            \pmb{v}(t) \\
            \pmb{\omega}(t)\times\pmb{R}(t) \\
            \pmb{f}(t) \\
            \pmb{\tau}(t) \\
        \end{array}
    \right)
\end{equation}

Then, we can evaluate Equation \ref{tau_ev} as follows:
\begin{equation}
    \begin{aligned}
        \pmb{\tau} &= \frac{\dif}{\dif t}(\pmb{I}\pmb{\omega}) \\
        &= \pmb{I}\dot{\pmb{\omega}} + \dot{\pmb{I}}\pmb{\omega} \\
        &= \pmb{I}\dot{\pmb{\omega}} + \frac{\dif}{\dif t}(\pmb{R}\pmb{I}_{body}\pmb{R}^{T})\pmb{\omega} \\
        &= \pmb{I}\dot{\pmb{\omega}} + (\dot{\pmb{R}}\pmb{I}_{body}\pmb{R}^{T} + \pmb{R}\pmb{I}_{body}\dot{\pmb{R}}^{T})\pmb{\omega} \\
        &= \pmb{I}\dot{\pmb{\omega}} + ([\pmb{\omega}]\pmb{R}\pmb{I}_{body}\pmb{R}^{T} + \pmb{R}\pmb{I}_{body}\pmb{R}^{T} \hat{\pmb{\omega}})\pmb{\omega} \\
        &= \pmb{I}\dot{\pmb{\omega}} + [\pmb{\omega}]\pmb{I}\pmb{\omega} - \pmb{I}[\pmb{\omega}]\pmb{\omega}
    \end{aligned}
    \label{tau_ev}
\end{equation}

Since $\pmb{\omega} \times \pmb{\omega}$ is zero, the final term can be cancels out. This relationship left is knowned as :
\begin{equation}
    \pmb{\tau} = \pmb{I}\dot{\pmb{\omega}} + [\pmb{\omega}]\pmb{I}\pmb{\omega}
    \label{tau}
\end{equation}

\subsection{Impulse}
When two bodies collide, those bodies and any other bodies, they are touching, experience very high forces of very short duration. In the case of ideal rigid bodies, the force magnitudes become infinite and the duration becomes infinitesimal. These forces are referred to as impulsive forces or shocks. One can see from Equation \ref{force} that shocks cause infinite accelerations, which makes direct numerical integration of the Newton–Euler equations impossible. One way to deal with this problem during simulation is to use a standard integration method up to the time of impact, then use an impulse–momentum law to determine the jump discontinuities in the velocities, and finally restart the integrator. \\

Let $[t, t+\Delta t]$ be a time step during which a collision occurs. Further, define $\mathbf{p}=\int_{t}^{t+\Delta t}\dif t$ as the impulse of the net force and $m\pmb{v}$ as translational momentum. Integrating Equation \ref{force} from $t$ to $t+\Delta t$ yields $m(\pmb{v}(t+\Delta t)-\pmb{v}(t))=\int_{t}^{t+\Delta t}\pmb{f}\dif t$, which states that $\Delta t$ impulse of the net applied force equals the change of translational momentum of the body. In rigid body collisions, $\Delta t$ approaches zero. Taking the limit as $\Delta t$ goes to zero, one obtains an impulse momentum law that is applied at the instant of impact to compute post-collision velocities. Since $\Delta t$ goes to zero and the velocities remain finite, the generalized position of the bodies are fixed during the impact. After processing the collision, one has the values of the generalized positions and velocities, which are the initial conditions to restart the integrator. Note that integration of the rotational Equation \ref{tau} yields an impulse–momentum law for determining jump discontinuities in the rotational velocities.

\subsection{Twist/Wrench}
We will now introduce vectors called twists, which describe velocities, and wrenches, which describe forces, and explain how these objects transform from one coordinate frame to another one. 
    \subsubsection{Twist} 
        A twist is a vector that expresses rigid motion or velocity. In Section 2.2, we saw how to parameterize the velocity of a rigid body as a linear velocity vector and an angular velocity vector. The coordinates of a twist are given as a 4-vector in $2$-D simulation, which we can check in \ref{twist}
            \begin{equation}
                \mathbf{v} = \left( \begin{array}{c} \pmb{\omega} \\ \pmb{v} \\ \end{array} \right )
                \label{twist}
            \end{equation}.
        The defination can be found in \ref{twist}, containing a linear velocity vector \(\pmb{v}\) and an angular velocity \(\pmb{\omega}\). According to 

    \subsubsection{Wrench}
        A wrench is a vector that expresses force and torque acting on a body. A wrench can be defined by
        \begin{equation}
            \mathbf{f} = \left( \begin{array}{c} \pmb{\tau} \\ \pmb{f} \\ \end{array} \right )
            \label{wrench} 
        \end{equation}

        A wrench contains an angular component $\pmb{\tau}$ and a linear component $f$, which are applied at the origin of the coordinate frame they are specified in.


\subsection{Newton-Euler Equation}
    The Newton-Euler equations for a rigid body can now be written in terms of the body's acceleration twist $\mathbf{v}$ methioned in \ref{twist} and the wrench $\mathbf{f}$ metioned in \ref{wrench} acting on the body. We can simply write the Newton and Eular equations,

    \begin{equation}
        \left( \begin{array}{c} \pmb{\tau} - \pmb{\omega} \times \pmb{I} \pmb{\omega}\\ \pmb{f} \end{array}\right) = \left( \begin{array}{cc} \pmb{I} & \pmb{0} \\ \pmb{0}& m\pmb{1}_{d\times d}\end{array} \right ) \dot{\mathbf{v}}
    \end{equation}
    $d$ stands for the number pf dimensions, like $d=2$ in $2$-D simulation. \\

    If we define $\mathcal{M}$ and $\mathbf{h}$ as Equation \ref{M} and \ref{h}

    \begin{equation}
        \mathcal{M} = \left( \begin{array}{cc} \pmb{I} & \pmb{0} \\ \pmb{0} & m\pmb{1}_{d \times d}\end{array} \right)
        \label{M}
    \end{equation}

    \begin{equation}
        \mathbf{h} = \left( \begin{array}{c} \pmb{\tau} - \pmb{\omega} \times \pmb{I} \pmb{\omega}\\ \pmb{f} \end{array}\right)
    \end{equation}
    So, we can rewrite Newton-Euler equation as,
    \begin{equation}
        \mathcal{M}\dot{\mathbf{v}} = \mathbf{h}
    \end{equation}


\section{Constrained Dynamics}
    most interesting simulations of rigid bodies involve some kind of constraints. Usually we want to model systems of bodies that are interacting in some way. Some bodies may be in contact with each other, or attached together by some types of joint. Since this report mainly research contact forces, we 
    \subsection{Expressing Constraints as Equations}
        We express constraints mathematically as algebraic matrix equations with position, velocity, or acceleration vectors as the unknowns. In general, the configuration space of a set of n rigid bodies has dimension 6n. Adding constraint equations restricts the position (or velocity, or acceleration) to a subspace of smaller dimension.\\

        The constraints discussed in this thesis will all be holonomic constraints, which means the constraint on the velocity can be found by taking the derivative of a position constraint. Constraints that do not fit this description are called nonholonomic. An example of a nonholonomic constraint is a ball rolling on a table without slipping. The position of the ball has five degrees of freedom, so there is only one degree of constraint (only the height is constrained). Taking the derivative of the position constraint would only give a velocity constraint of degree one. Additional constraints on the velocity are needed to prevent the ball from slipping.\\

        We will describe position constraints with a constraint function $g(\pmb{p})$, which is a function from the space of possible positions of the rigid bodies, to $\mathbf{R}^d$, where $d$ is the number of degrees of freedom that constraint removes from the dynamic system. If the constraint function returns a zero vector, then the position $\pmb{p}$ satifies the constraint. \\

        Position constraint can be divided into \textit{equality constraint}(e.g., joint constraints), where the constraint is $g(\pmb{p}) = 0$, and \textit{inequality constraints}(e.g., contact constraints), where $g(\pmb{p}) \ge 0$. \\

        Constraining the position of an object also constrains its velocity. The velocity constraint function can be found by taking the time derivative of the constraint function. We want the velocity constraint function to be a linear function of the twists representing the velocities of all of the rigid bodies in the system. We let $\mathbf{v}$ to be:
        \begin{equation}
            \mathbf{v} = \left( \begin{array}{c}\mathbf{v}_1 \\\mathbf{v}_2 \\ ... \\ \mathbf{v}_n \end{array}\right)
        \end{equation}
        Velocity constraints are of the form
        \begin{equation}
            \frac{\dif g_i}{\dif t} = J_i \mathbf{v} = (\mathbf{J}_{i1} ,...,\mathbf{J}_{in} )\left(\begin{array}{c}\left(\begin{array}{c}\pmb{\omega}_1 \\ \pmb{v}_1\end{array}\right) \\ \left(\begin{array}{c}\pmb{\omega}_2 \\ \pmb{v}_2\end{array}\right) \\ ... \\ \left(\begin{array}{c}\pmb{\omega}_n \\ \pmb{v}_n\end{array}\right)\end{array}\right)
        \end{equation}
        where $i$ is unique for each constraint. The matrix $J$ is called the constraint's \textit{Jacobian matrix}, which we will refer to simply as the Jacobian. \\

        When there are many constraints, like a system with $n$ bodies, $m$ constraints, the velocity constraint equation looks like
        \begin{equation}
            \begin{aligned}
            \frac{\dif g}{\dif t} &= \left( \begin{array}{c} g_1 \\g_2 \\ ...\\g_m\end{array}\right)= \mathcal{J}\mathbf{v} = \left( \begin{array}{c} J_1 \\ ... \\ J_n \end{array} \right) \left( \begin{array}{c} \mathbf{v}_1 \\ ... \\ \mathbf{v}_n \end{array} \right) \\
            &= \left( \begin{array}{cccc} \pmb{J}_{11} & \pmb{J}_{12} & ... & \pmb{J}_{1n} \\ \pmb{J}_{21} & \pmb{J}_{22} & ... & \pmb{J}_{2n} \\ ... & ... & ... & ... \\ \pmb{J}_{m1} & \pmb{J}_{m2} & ... & \pmb{J}_{nn}\end{array} \right) \left(\begin{array}{c}\left(\begin{array}{c}\pmb{\omega}_1 \\ \pmb{v}_1\end{array}\right) \\ \left(\begin{array}{c}\pmb{\omega}_2 \\ \pmb{v}_2\end{array}\right) \\ ... \\ \left(\begin{array}{c}\pmb{\omega}_n \\ \pmb{v}_n\end{array}\right)\end{array}\right)
            \end{aligned} 
        \end{equation}
        Notes:
        \begin{itemize}
            \item $\pmb{J}$ is used for Jacobians that relate the velocity of a single body to a single constraint.
            \item $J$ is used for Jacobians that relate the velocity of mant bodies with many bodies to a single constraint.
            \item $\mathcal{J}$ is used for Jacobians that relate velocity of many bodies to many constraints. 
        \end{itemize}

    \subsection{Solving Constrained Dynamics Equations using Largange Multipliers}
        A constraint Jacobian matrix $\mathcal{J}$ has a dual use. In addition to relating the vlocities to the rate of change of the constraint function $\pmb{g}$, the rows of $\mathcal{J}$ act as basis vectors for constraint forces. Thus, actually we do just need to solve for the coefficient vector $\pmb{\lambda}$, which is called the \textit{Lagrange multipliers} that contains the magnitudes of the forces that correspond to each of these basis vectors.

        \subsubsection{Solving Dynamics at the Force-Acceleration Level}
        Combining the Newton-Euler equations,
        \begin{equation}
            \mathcal{M}\dot{\mathbf{v}} = \mathcal{J}^{T}\pmb{\lambda} + \mathbf{h}_{ext}
        \end{equation}
        where $\mathcal{M}$ was defined in Equation \ref{M}, $\mathbf{h}_{ext}$ holds external and gyroscopic force terms. 

    \subsection{Contact Constraints}
        From other papers \cite{bender2014interactive}, we can conclude some import features of contact constraint:
        \begin{itemize}
            \item Contact forces can push bodies apart, but cannot pull bodies towards each other. This leads to inequalities in the constraint equations.
            \item Contact are transient - they come and go. Contacts can appear suddenly when bodies are moving towards each other, resulting in an impact.
        \end{itemize}
        Impacts between rigid bodies in reality result in large forces applied in over a very short time period, leading to a sudden changes in velocity. There are serveral methods for handling contact in rigid body simulations. Mirtich\cite{mirtich1998rigid} give a summary of serval different methods, and explains the advantages and disadvantages of each. We mainly introduced one  method called .... from Kenny\cite{erleben2017rigid}
        \begin{itemize}
            \item A contract consist of a pair of contact points, one point attached, one point attached to one rigid body, and the other point attached to another rigid body. The contact points are sufficiently close together for our collision detection algorithm to report a collision. Normally, we caculate the distance between two rigid bodies(mainly on circle shape).
            \begin{equation}
                g_i = \|\pmb{p}_{i1} - \pmb{p}_{i2}\| - |r_{i1} + r_{i2}|
            \end{equation}
            \item A contact normal is a unit vector that is normal to one or both of the surfaces at the contact points.
            \item The contact constraint Jacobian for constraint $i$, denoted by $\pmb{J}_{c_i}$, is a $1\times 6n$ matrix, where n is the number of bodies (the subscript `$c$' here stands for `contact').
            \item A contact force $\pmb{J}_{c_i}^{T}\lambda _{c_i}$ is a force acting in the direction of the contact normal which prevents the two rigid bodies from interpenetrating.
            \item The separation distance of a contact is the normal component of the distance between the two contact points. It is negative when the bodies are interpenetrating at the contact. The constraint function $g_{i}(\pmb{p})$ for a contact constraint returns the separation distance. In other words, $g_i \ge 0$.
            \item The relative normal acceleration $a_i$ of a contact is the second derivative of the separation distance with respect to time ($a_i = \ddot{g}_i$). The acceleration constrait is satisfied when $a_i = \pmb{J}_{c_i} \dot{\mathbf{v}}_i+ k_i \ge 0$
        \end{itemize}
        When $g$ or $\dot{g}$ are positive, it indicates that the bodies are not touching, or moving apart at that contact, respectively, in other words, no acceleration is needed. The assumption that at each resting contact $g = 0$ and $\dot{g} = 0 $ are critical to the constraint that we enumerate below.\\

        Let $\pmb{a}$ be a vector containing the relative normal accelerations \{$a_1,...,a_n$\} for all contacts and $\pmb{\lambda}_c$ be a vector of contact force multipliers \{$\lambda_{c_1},..., \lambda_{c_n}$\}. The vectors $\pmb{a}$ and $\pmb{\lambda}_c$ are linearly related at a given $\mathbf{p}$. Additionally, we have three constraints:
        \begin{enumerate}
            \item The relative normal accelerations must be nongative:
            \begin{equation}
                \pmb{a} = \mathcal{J}_c\dot{\mathbf{v}}+\pmb{k} \ge 0
            \end{equation}
            Since $g$ and $\dot{g}$ are zero, a negative acceleration would cause interpenetration.
            \item The contact force magnitudes must be nonnegative (so as to push the bodies apart): 
            \begin{equation}
                \pmb{\lambda}_c \ge \pmb{0}
            \end{equation}
            \item For each contact $i$, at least one of $a_i, \lambda_{c_i}$ must be zero, we can write that
            \begin{equation}
                a_{i}^{T}\lambda_{c_{i}} = 0
            \end{equation}
            Since there can only be a contact force if the bodies are actually touching($g_i\lambda_{c_i} = 0$). Differentiating this twice using chain rule, we get $\ddot{g}_i\lambda_{c_i} + 2\dot{g}\dot{\lambda}_{c_i} + g_i\lambda_{i} = 0$. Since we assume $g_i = 0$ and $\dot{g}_i = 0$, the second and third term cancel out leaving simply:
            \begin{equation}
                \ddot{g}_i \lambda_{c_i}=0
            \end{equation}
            In other words:
            \begin{equation}
                \pmb{a}^{T}\pmb{\lambda}_{c} = \pmb{0}
            \end{equation}
        \end{enumerate}

        Then we can write the whole system equation as Equation \ref{}
        \begin{equation}
            \left(\begin{array}{c}\pmb{0} \\ \pmb{a}\end{array}\right)
             = \left(\begin{array}{cc} \mathcal{M} & - \mathcal{J}_{c}^{T} \\ \mathcal{J}_{c} & 0 \end{array}\right)\left(\begin{array}{c}\dot{\mathbf{v}} \\ \pmb{\lambda}_{c}\end{array}\right) - \left(\begin{array}{c}\mathbf{h}_{ext} \\ -\pmb{k}_{c}\end{array}\right)
             \label{sys}
        \end{equation}

        With the condition,
        \begin{equation}
            \begin{cases}
                \pmb{a} \ge 0 \\
                \pmb{\lambda}_{c} \ge 0 \\
                \pmb{a}^{T}\pmb{\lambda}_c = \pmb{0}
            \end{cases}
        \end{equation}

        \subsection{Linear Complementarity Problems(LCP)}
            A \textit{Linear Complementarity Problem}(LCP) is, given a $n \times n$ matrix $\pmb{M}$ and a $n$-vector $\pmb{q}$, the problem of finding values for the variables $\{z_1, z_2 ... z_n\}$ and $\{\omega_1, \omega_2, ... , \omega_n\}$ like,

            \begin{equation}
                \left(\begin{array}{c} \omega_1 \\ \omega_2 \\ ... \\ \omega_n \end{array} \right) = \pmb{M} \left(\begin{array}{c} z_1 \\ z_2 \\ ... \\ z_n \end{array}\right) + \pmb{q}
            \end{equation}

            and for all i from $1$ to $n$, \textit{complementarity constraints} can be,
            \begin{equation}
                \begin{cases}
                    z_i \ge 0 \\
                    \omega_i \ge 0 \\
                    z_i \omega_i = 0 \\
                \end{cases}
            \end{equation}


\section{Models for Systems with Frictional Contacts}

        The laws of physics must be combined into what we term an `instantaneous-time' model, which describes the continuous-time motions of the rigid bodies. Following this, we discretize this model over time to obtain a ‘discrete-time’ model, which is a sequence of time-stepping subproblems. The subproblems are formulated and numerically solved at every time step to simulate the system. \\

        It is well known that when friction is added to the acceleration-level dynamics equations (Equation \ref{sys}).Anitescu and Potra \cite{anitescu1997modeling} present a time-step method that combines the acceleration-level LCP with an intergration step for the velocities, arriving at a method having method having velocities and impulses as unknowns, rather tham acceleration and forces. the methud is guaranteed to have a solution, regardless of the number of contacts. \\

        To discretize the system \ref{sys}, the acceleration can be approximated by \cite{anitescu1997modeling} as:
            \begin{equation}
                \dot{\mathbf{v}} \approx \frac{(\mathbf{v}_{t+h} - \mathbf{v}_t)}{h}
            \end{equation}
        $\mathbf{v}_t$ and $\mathbf{v}_{t+h}$ are the velocities at the beginning of the current time step, and the next time step, $h$ is the time step. We then arrive at the mixed LCP

        \begin{equation}
            \left(\begin{array}{c}\pmb{0} \\ \pmb{a} \end{array}\right) = \left(\begin{array}{cc} \mathcal{M} & -\mathcal{J}_{c}^{T} \\ \mathcal{J}_{c} & 0 \end{array}\right) \left(\begin{array}{c} \mathbf{v}_{t+h} \\ \pmb{\lambda}_{c} \end{array}\right) + 
            \label{lcp}
        \end{equation}

    \subsection{Friction Constraints}
        Coulomb friction is intrdiced by adding some addtional forces and constraints to the LCP \ref{lcp} \\

        In true Coulomb friction, the magnitude of the friction force must be less than $\mu$ times the magnitude of the normal force:
        \begin{equation}
            \|\pmb{f}_f\| \le \mu\|\pmb{f}_n\|
        \end{equation}

        Geometrically, the contact forces is restricted to lie inside of a cone. However, to fit friction into our linear algebraic framework, it helps to approximate the friction cone with a ployhedral cone. We do this by choosing a set of direction vectors \(\pmb{d}_{i1}, \pmb{d}_{i2}...\pmb{d}_{in}\). The expected frictional impulse for contact $i$ as:
        \begin{equation}
            J_{f_i}^{T}\pmb{\lambda}_{f_i} = (\begin{array}{cccc} \pmb{d}_{i1} & \pmb{d}_{i2} & ... & \pmb{d}_{in} \end{array})\left(\begin{array}{c} \lambda_{f_{i1}} \\ \lambda_{f_{i2}} \\ ... \\ \lambda_{f_{in}}\end{array}\right)
        \end{equation}


    \subsection{Projected Gauss–Seidel(PGS) solver for contact}

      