\chapter{Implementation Details}

This chapter mainly talk about 
\begin{enumerate}
    \item Do the simulation based one computer physics library. Totally, 100 different rigid motion simulation should be finished. For every simulation, fixed steps should be recorded.
    \item Restore information of each state of per simulation by using \textbf{XML} formats, including positions, velocities, contact forces, etc.
    \item Read \textbf{XML} file, and then generate some grid images for training.
    \item Do the deep learning based on training dataset which is created by last step.
    \item Apply the trained model to initialize the values of contact forces($\lambda$). Then compare deep learning method and classical methods.
\end{enumerate}

\section{Rigid Motion Simulation}

I chose \textbf{pybox2d}\footnote{\url{https://github.com/pybox2d/pybox2d}} as the main physics engine to implemente computer simulation. \textbf{pybox2d} is a 2D physics library for your games and simple simulations. It's based on the Box2D library, written in \textit{C++}. It supports several shape types (circle, polygon, thin line segments), and quite a few joint types (revolute, prismatic, wheel, etc.).

\subsection{Simulation Configuration}

\subsection{Simulation Details}


\section{Data Generation}

\subsection{XML Restoration}

An example for one body information is given below.
\begin{lstlisting}[language=XML]
    <body index="86" type="free">
        <mass value="3.14159274101"/>
        <position x="7.79289388657" y="2.62924313545"/>
        <velocity x="2.7878344059" y="-1.45545887947"/>
        <orientation theta="-0.115291565657"/>
        <inertia value="1.57079637051"/>
        <spin omega="-2.33787894249"/>
        <shape value="circle"/>
    </body>
    ...
\end{lstlisting}

Another example \textit{XML} code is for contatc force.
\begin{lstlisting}[language=XML]
    <contact index="1" master="2" master_shape="b2CircleShape(childCount=1, pos=b2Vec2(0,0), radius=1.2000000476837158, type=0,)" slave="97" slave_shape="b2CircleShape(childCount=1, pos=b2Vec2(0,0), radius=1.2000000476837158, type=0, )">
        <position x="0.21963849663734436" y="13.875240325927734"/>
        <normal normal="b2Vec2(-1,2.9819e-05)"/>
        <impulse n="0.005236322991549969" t="-0.002184529323130846"/>
    </contact>
    ...
\end{lstlisting}

\subsection{SPH configuration}
\subsection{XML to grid}
After each state has been stored in \textit{XML} file, then 


\section{CNN Training}

\subsection{CNN  Structure Design}

\subsection{Traing Configuration}
\subsubsection{Loss Function}
Firstly, we define a filter funtion,
\begin{equation}
    f(x)= \begin{cases} 0, \quad x = 0 \\ 1, \quad x \ne 0 \end{cases}
\end{equation}
Then, we can update the loss function based on Euqation \ref{eq:mse}

\begin{equation}
    L()
\end{equation}

\subsection{Training Details}

\section{Simualtion based on Trained Model}
